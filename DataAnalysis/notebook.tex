
% Default to the notebook output style

    


% Inherit from the specified cell style.




    
\documentclass[11pt]{article}

    
    
    \usepackage[T1]{fontenc}
    % Nicer default font (+ math font) than Computer Modern for most use cases
    \usepackage{mathpazo}

    % Basic figure setup, for now with no caption control since it's done
    % automatically by Pandoc (which extracts ![](path) syntax from Markdown).
    \usepackage{graphicx}
    % We will generate all images so they have a width \maxwidth. This means
    % that they will get their normal width if they fit onto the page, but
    % are scaled down if they would overflow the margins.
    \makeatletter
    \def\maxwidth{\ifdim\Gin@nat@width>\linewidth\linewidth
    \else\Gin@nat@width\fi}
    \makeatother
    \let\Oldincludegraphics\includegraphics
    % Set max figure width to be 80% of text width, for now hardcoded.
    \renewcommand{\includegraphics}[1]{\Oldincludegraphics[width=.8\maxwidth]{#1}}
    % Ensure that by default, figures have no caption (until we provide a
    % proper Figure object with a Caption API and a way to capture that
    % in the conversion process - todo).
    \usepackage{caption}
    \DeclareCaptionLabelFormat{nolabel}{}
    \captionsetup{labelformat=nolabel}

    \usepackage{adjustbox} % Used to constrain images to a maximum size 
    \usepackage{xcolor} % Allow colors to be defined
    \usepackage{enumerate} % Needed for markdown enumerations to work
    \usepackage{geometry} % Used to adjust the document margins
    \usepackage{amsmath} % Equations
    \usepackage{amssymb} % Equations
    \usepackage{textcomp} % defines textquotesingle
    % Hack from http://tex.stackexchange.com/a/47451/13684:
    \AtBeginDocument{%
        \def\PYZsq{\textquotesingle}% Upright quotes in Pygmentized code
    }
    \usepackage{upquote} % Upright quotes for verbatim code
    \usepackage{eurosym} % defines \euro
    \usepackage[mathletters]{ucs} % Extended unicode (utf-8) support
    \usepackage[utf8x]{inputenc} % Allow utf-8 characters in the tex document
    \usepackage{fancyvrb} % verbatim replacement that allows latex
    \usepackage{grffile} % extends the file name processing of package graphics 
                         % to support a larger range 
    % The hyperref package gives us a pdf with properly built
    % internal navigation ('pdf bookmarks' for the table of contents,
    % internal cross-reference links, web links for URLs, etc.)
    \usepackage{hyperref}
    \usepackage{longtable} % longtable support required by pandoc >1.10
    \usepackage{booktabs}  % table support for pandoc > 1.12.2
    \usepackage[inline]{enumitem} % IRkernel/repr support (it uses the enumerate* environment)
    \usepackage[normalem]{ulem} % ulem is needed to support strikethroughs (\sout)
                                % normalem makes italics be italics, not underlines
    

    
    
    % Colors for the hyperref package
    \definecolor{urlcolor}{rgb}{0,.145,.698}
    \definecolor{linkcolor}{rgb}{.71,0.21,0.01}
    \definecolor{citecolor}{rgb}{.12,.54,.11}

    % ANSI colors
    \definecolor{ansi-black}{HTML}{3E424D}
    \definecolor{ansi-black-intense}{HTML}{282C36}
    \definecolor{ansi-red}{HTML}{E75C58}
    \definecolor{ansi-red-intense}{HTML}{B22B31}
    \definecolor{ansi-green}{HTML}{00A250}
    \definecolor{ansi-green-intense}{HTML}{007427}
    \definecolor{ansi-yellow}{HTML}{DDB62B}
    \definecolor{ansi-yellow-intense}{HTML}{B27D12}
    \definecolor{ansi-blue}{HTML}{208FFB}
    \definecolor{ansi-blue-intense}{HTML}{0065CA}
    \definecolor{ansi-magenta}{HTML}{D160C4}
    \definecolor{ansi-magenta-intense}{HTML}{A03196}
    \definecolor{ansi-cyan}{HTML}{60C6C8}
    \definecolor{ansi-cyan-intense}{HTML}{258F8F}
    \definecolor{ansi-white}{HTML}{C5C1B4}
    \definecolor{ansi-white-intense}{HTML}{A1A6B2}

    % commands and environments needed by pandoc snippets
    % extracted from the output of `pandoc -s`
    \providecommand{\tightlist}{%
      \setlength{\itemsep}{0pt}\setlength{\parskip}{0pt}}
    \DefineVerbatimEnvironment{Highlighting}{Verbatim}{commandchars=\\\{\}}
    % Add ',fontsize=\small' for more characters per line
    \newenvironment{Shaded}{}{}
    \newcommand{\KeywordTok}[1]{\textcolor[rgb]{0.00,0.44,0.13}{\textbf{{#1}}}}
    \newcommand{\DataTypeTok}[1]{\textcolor[rgb]{0.56,0.13,0.00}{{#1}}}
    \newcommand{\DecValTok}[1]{\textcolor[rgb]{0.25,0.63,0.44}{{#1}}}
    \newcommand{\BaseNTok}[1]{\textcolor[rgb]{0.25,0.63,0.44}{{#1}}}
    \newcommand{\FloatTok}[1]{\textcolor[rgb]{0.25,0.63,0.44}{{#1}}}
    \newcommand{\CharTok}[1]{\textcolor[rgb]{0.25,0.44,0.63}{{#1}}}
    \newcommand{\StringTok}[1]{\textcolor[rgb]{0.25,0.44,0.63}{{#1}}}
    \newcommand{\CommentTok}[1]{\textcolor[rgb]{0.38,0.63,0.69}{\textit{{#1}}}}
    \newcommand{\OtherTok}[1]{\textcolor[rgb]{0.00,0.44,0.13}{{#1}}}
    \newcommand{\AlertTok}[1]{\textcolor[rgb]{1.00,0.00,0.00}{\textbf{{#1}}}}
    \newcommand{\FunctionTok}[1]{\textcolor[rgb]{0.02,0.16,0.49}{{#1}}}
    \newcommand{\RegionMarkerTok}[1]{{#1}}
    \newcommand{\ErrorTok}[1]{\textcolor[rgb]{1.00,0.00,0.00}{\textbf{{#1}}}}
    \newcommand{\NormalTok}[1]{{#1}}
    
    % Additional commands for more recent versions of Pandoc
    \newcommand{\ConstantTok}[1]{\textcolor[rgb]{0.53,0.00,0.00}{{#1}}}
    \newcommand{\SpecialCharTok}[1]{\textcolor[rgb]{0.25,0.44,0.63}{{#1}}}
    \newcommand{\VerbatimStringTok}[1]{\textcolor[rgb]{0.25,0.44,0.63}{{#1}}}
    \newcommand{\SpecialStringTok}[1]{\textcolor[rgb]{0.73,0.40,0.53}{{#1}}}
    \newcommand{\ImportTok}[1]{{#1}}
    \newcommand{\DocumentationTok}[1]{\textcolor[rgb]{0.73,0.13,0.13}{\textit{{#1}}}}
    \newcommand{\AnnotationTok}[1]{\textcolor[rgb]{0.38,0.63,0.69}{\textbf{\textit{{#1}}}}}
    \newcommand{\CommentVarTok}[1]{\textcolor[rgb]{0.38,0.63,0.69}{\textbf{\textit{{#1}}}}}
    \newcommand{\VariableTok}[1]{\textcolor[rgb]{0.10,0.09,0.49}{{#1}}}
    \newcommand{\ControlFlowTok}[1]{\textcolor[rgb]{0.00,0.44,0.13}{\textbf{{#1}}}}
    \newcommand{\OperatorTok}[1]{\textcolor[rgb]{0.40,0.40,0.40}{{#1}}}
    \newcommand{\BuiltInTok}[1]{{#1}}
    \newcommand{\ExtensionTok}[1]{{#1}}
    \newcommand{\PreprocessorTok}[1]{\textcolor[rgb]{0.74,0.48,0.00}{{#1}}}
    \newcommand{\AttributeTok}[1]{\textcolor[rgb]{0.49,0.56,0.16}{{#1}}}
    \newcommand{\InformationTok}[1]{\textcolor[rgb]{0.38,0.63,0.69}{\textbf{\textit{{#1}}}}}
    \newcommand{\WarningTok}[1]{\textcolor[rgb]{0.38,0.63,0.69}{\textbf{\textit{{#1}}}}}
    
    
    % Define a nice break command that doesn't care if a line doesn't already
    % exist.
    \def\br{\hspace*{\fill} \\* }
    % Math Jax compatability definitions
    \def\gt{>}
    \def\lt{<}
    % Document parameters
    \title{web Scraping}
    
    
    

    % Pygments definitions
    
\makeatletter
\def\PY@reset{\let\PY@it=\relax \let\PY@bf=\relax%
    \let\PY@ul=\relax \let\PY@tc=\relax%
    \let\PY@bc=\relax \let\PY@ff=\relax}
\def\PY@tok#1{\csname PY@tok@#1\endcsname}
\def\PY@toks#1+{\ifx\relax#1\empty\else%
    \PY@tok{#1}\expandafter\PY@toks\fi}
\def\PY@do#1{\PY@bc{\PY@tc{\PY@ul{%
    \PY@it{\PY@bf{\PY@ff{#1}}}}}}}
\def\PY#1#2{\PY@reset\PY@toks#1+\relax+\PY@do{#2}}

\expandafter\def\csname PY@tok@w\endcsname{\def\PY@tc##1{\textcolor[rgb]{0.73,0.73,0.73}{##1}}}
\expandafter\def\csname PY@tok@c\endcsname{\let\PY@it=\textit\def\PY@tc##1{\textcolor[rgb]{0.25,0.50,0.50}{##1}}}
\expandafter\def\csname PY@tok@cp\endcsname{\def\PY@tc##1{\textcolor[rgb]{0.74,0.48,0.00}{##1}}}
\expandafter\def\csname PY@tok@k\endcsname{\let\PY@bf=\textbf\def\PY@tc##1{\textcolor[rgb]{0.00,0.50,0.00}{##1}}}
\expandafter\def\csname PY@tok@kp\endcsname{\def\PY@tc##1{\textcolor[rgb]{0.00,0.50,0.00}{##1}}}
\expandafter\def\csname PY@tok@kt\endcsname{\def\PY@tc##1{\textcolor[rgb]{0.69,0.00,0.25}{##1}}}
\expandafter\def\csname PY@tok@o\endcsname{\def\PY@tc##1{\textcolor[rgb]{0.40,0.40,0.40}{##1}}}
\expandafter\def\csname PY@tok@ow\endcsname{\let\PY@bf=\textbf\def\PY@tc##1{\textcolor[rgb]{0.67,0.13,1.00}{##1}}}
\expandafter\def\csname PY@tok@nb\endcsname{\def\PY@tc##1{\textcolor[rgb]{0.00,0.50,0.00}{##1}}}
\expandafter\def\csname PY@tok@nf\endcsname{\def\PY@tc##1{\textcolor[rgb]{0.00,0.00,1.00}{##1}}}
\expandafter\def\csname PY@tok@nc\endcsname{\let\PY@bf=\textbf\def\PY@tc##1{\textcolor[rgb]{0.00,0.00,1.00}{##1}}}
\expandafter\def\csname PY@tok@nn\endcsname{\let\PY@bf=\textbf\def\PY@tc##1{\textcolor[rgb]{0.00,0.00,1.00}{##1}}}
\expandafter\def\csname PY@tok@ne\endcsname{\let\PY@bf=\textbf\def\PY@tc##1{\textcolor[rgb]{0.82,0.25,0.23}{##1}}}
\expandafter\def\csname PY@tok@nv\endcsname{\def\PY@tc##1{\textcolor[rgb]{0.10,0.09,0.49}{##1}}}
\expandafter\def\csname PY@tok@no\endcsname{\def\PY@tc##1{\textcolor[rgb]{0.53,0.00,0.00}{##1}}}
\expandafter\def\csname PY@tok@nl\endcsname{\def\PY@tc##1{\textcolor[rgb]{0.63,0.63,0.00}{##1}}}
\expandafter\def\csname PY@tok@ni\endcsname{\let\PY@bf=\textbf\def\PY@tc##1{\textcolor[rgb]{0.60,0.60,0.60}{##1}}}
\expandafter\def\csname PY@tok@na\endcsname{\def\PY@tc##1{\textcolor[rgb]{0.49,0.56,0.16}{##1}}}
\expandafter\def\csname PY@tok@nt\endcsname{\let\PY@bf=\textbf\def\PY@tc##1{\textcolor[rgb]{0.00,0.50,0.00}{##1}}}
\expandafter\def\csname PY@tok@nd\endcsname{\def\PY@tc##1{\textcolor[rgb]{0.67,0.13,1.00}{##1}}}
\expandafter\def\csname PY@tok@s\endcsname{\def\PY@tc##1{\textcolor[rgb]{0.73,0.13,0.13}{##1}}}
\expandafter\def\csname PY@tok@sd\endcsname{\let\PY@it=\textit\def\PY@tc##1{\textcolor[rgb]{0.73,0.13,0.13}{##1}}}
\expandafter\def\csname PY@tok@si\endcsname{\let\PY@bf=\textbf\def\PY@tc##1{\textcolor[rgb]{0.73,0.40,0.53}{##1}}}
\expandafter\def\csname PY@tok@se\endcsname{\let\PY@bf=\textbf\def\PY@tc##1{\textcolor[rgb]{0.73,0.40,0.13}{##1}}}
\expandafter\def\csname PY@tok@sr\endcsname{\def\PY@tc##1{\textcolor[rgb]{0.73,0.40,0.53}{##1}}}
\expandafter\def\csname PY@tok@ss\endcsname{\def\PY@tc##1{\textcolor[rgb]{0.10,0.09,0.49}{##1}}}
\expandafter\def\csname PY@tok@sx\endcsname{\def\PY@tc##1{\textcolor[rgb]{0.00,0.50,0.00}{##1}}}
\expandafter\def\csname PY@tok@m\endcsname{\def\PY@tc##1{\textcolor[rgb]{0.40,0.40,0.40}{##1}}}
\expandafter\def\csname PY@tok@gh\endcsname{\let\PY@bf=\textbf\def\PY@tc##1{\textcolor[rgb]{0.00,0.00,0.50}{##1}}}
\expandafter\def\csname PY@tok@gu\endcsname{\let\PY@bf=\textbf\def\PY@tc##1{\textcolor[rgb]{0.50,0.00,0.50}{##1}}}
\expandafter\def\csname PY@tok@gd\endcsname{\def\PY@tc##1{\textcolor[rgb]{0.63,0.00,0.00}{##1}}}
\expandafter\def\csname PY@tok@gi\endcsname{\def\PY@tc##1{\textcolor[rgb]{0.00,0.63,0.00}{##1}}}
\expandafter\def\csname PY@tok@gr\endcsname{\def\PY@tc##1{\textcolor[rgb]{1.00,0.00,0.00}{##1}}}
\expandafter\def\csname PY@tok@ge\endcsname{\let\PY@it=\textit}
\expandafter\def\csname PY@tok@gs\endcsname{\let\PY@bf=\textbf}
\expandafter\def\csname PY@tok@gp\endcsname{\let\PY@bf=\textbf\def\PY@tc##1{\textcolor[rgb]{0.00,0.00,0.50}{##1}}}
\expandafter\def\csname PY@tok@go\endcsname{\def\PY@tc##1{\textcolor[rgb]{0.53,0.53,0.53}{##1}}}
\expandafter\def\csname PY@tok@gt\endcsname{\def\PY@tc##1{\textcolor[rgb]{0.00,0.27,0.87}{##1}}}
\expandafter\def\csname PY@tok@err\endcsname{\def\PY@bc##1{\setlength{\fboxsep}{0pt}\fcolorbox[rgb]{1.00,0.00,0.00}{1,1,1}{\strut ##1}}}
\expandafter\def\csname PY@tok@kc\endcsname{\let\PY@bf=\textbf\def\PY@tc##1{\textcolor[rgb]{0.00,0.50,0.00}{##1}}}
\expandafter\def\csname PY@tok@kd\endcsname{\let\PY@bf=\textbf\def\PY@tc##1{\textcolor[rgb]{0.00,0.50,0.00}{##1}}}
\expandafter\def\csname PY@tok@kn\endcsname{\let\PY@bf=\textbf\def\PY@tc##1{\textcolor[rgb]{0.00,0.50,0.00}{##1}}}
\expandafter\def\csname PY@tok@kr\endcsname{\let\PY@bf=\textbf\def\PY@tc##1{\textcolor[rgb]{0.00,0.50,0.00}{##1}}}
\expandafter\def\csname PY@tok@bp\endcsname{\def\PY@tc##1{\textcolor[rgb]{0.00,0.50,0.00}{##1}}}
\expandafter\def\csname PY@tok@fm\endcsname{\def\PY@tc##1{\textcolor[rgb]{0.00,0.00,1.00}{##1}}}
\expandafter\def\csname PY@tok@vc\endcsname{\def\PY@tc##1{\textcolor[rgb]{0.10,0.09,0.49}{##1}}}
\expandafter\def\csname PY@tok@vg\endcsname{\def\PY@tc##1{\textcolor[rgb]{0.10,0.09,0.49}{##1}}}
\expandafter\def\csname PY@tok@vi\endcsname{\def\PY@tc##1{\textcolor[rgb]{0.10,0.09,0.49}{##1}}}
\expandafter\def\csname PY@tok@vm\endcsname{\def\PY@tc##1{\textcolor[rgb]{0.10,0.09,0.49}{##1}}}
\expandafter\def\csname PY@tok@sa\endcsname{\def\PY@tc##1{\textcolor[rgb]{0.73,0.13,0.13}{##1}}}
\expandafter\def\csname PY@tok@sb\endcsname{\def\PY@tc##1{\textcolor[rgb]{0.73,0.13,0.13}{##1}}}
\expandafter\def\csname PY@tok@sc\endcsname{\def\PY@tc##1{\textcolor[rgb]{0.73,0.13,0.13}{##1}}}
\expandafter\def\csname PY@tok@dl\endcsname{\def\PY@tc##1{\textcolor[rgb]{0.73,0.13,0.13}{##1}}}
\expandafter\def\csname PY@tok@s2\endcsname{\def\PY@tc##1{\textcolor[rgb]{0.73,0.13,0.13}{##1}}}
\expandafter\def\csname PY@tok@sh\endcsname{\def\PY@tc##1{\textcolor[rgb]{0.73,0.13,0.13}{##1}}}
\expandafter\def\csname PY@tok@s1\endcsname{\def\PY@tc##1{\textcolor[rgb]{0.73,0.13,0.13}{##1}}}
\expandafter\def\csname PY@tok@mb\endcsname{\def\PY@tc##1{\textcolor[rgb]{0.40,0.40,0.40}{##1}}}
\expandafter\def\csname PY@tok@mf\endcsname{\def\PY@tc##1{\textcolor[rgb]{0.40,0.40,0.40}{##1}}}
\expandafter\def\csname PY@tok@mh\endcsname{\def\PY@tc##1{\textcolor[rgb]{0.40,0.40,0.40}{##1}}}
\expandafter\def\csname PY@tok@mi\endcsname{\def\PY@tc##1{\textcolor[rgb]{0.40,0.40,0.40}{##1}}}
\expandafter\def\csname PY@tok@il\endcsname{\def\PY@tc##1{\textcolor[rgb]{0.40,0.40,0.40}{##1}}}
\expandafter\def\csname PY@tok@mo\endcsname{\def\PY@tc##1{\textcolor[rgb]{0.40,0.40,0.40}{##1}}}
\expandafter\def\csname PY@tok@ch\endcsname{\let\PY@it=\textit\def\PY@tc##1{\textcolor[rgb]{0.25,0.50,0.50}{##1}}}
\expandafter\def\csname PY@tok@cm\endcsname{\let\PY@it=\textit\def\PY@tc##1{\textcolor[rgb]{0.25,0.50,0.50}{##1}}}
\expandafter\def\csname PY@tok@cpf\endcsname{\let\PY@it=\textit\def\PY@tc##1{\textcolor[rgb]{0.25,0.50,0.50}{##1}}}
\expandafter\def\csname PY@tok@c1\endcsname{\let\PY@it=\textit\def\PY@tc##1{\textcolor[rgb]{0.25,0.50,0.50}{##1}}}
\expandafter\def\csname PY@tok@cs\endcsname{\let\PY@it=\textit\def\PY@tc##1{\textcolor[rgb]{0.25,0.50,0.50}{##1}}}

\def\PYZbs{\char`\\}
\def\PYZus{\char`\_}
\def\PYZob{\char`\{}
\def\PYZcb{\char`\}}
\def\PYZca{\char`\^}
\def\PYZam{\char`\&}
\def\PYZlt{\char`\<}
\def\PYZgt{\char`\>}
\def\PYZsh{\char`\#}
\def\PYZpc{\char`\%}
\def\PYZdl{\char`\$}
\def\PYZhy{\char`\-}
\def\PYZsq{\char`\'}
\def\PYZdq{\char`\"}
\def\PYZti{\char`\~}
% for compatibility with earlier versions
\def\PYZat{@}
\def\PYZlb{[}
\def\PYZrb{]}
\makeatother


    % Exact colors from NB
    \definecolor{incolor}{rgb}{0.0, 0.0, 0.5}
    \definecolor{outcolor}{rgb}{0.545, 0.0, 0.0}



    
    % Prevent overflowing lines due to hard-to-break entities
    \sloppy 
    % Setup hyperref package
    \hypersetup{
      breaklinks=true,  % so long urls are correctly broken across lines
      colorlinks=true,
      urlcolor=urlcolor,
      linkcolor=linkcolor,
      citecolor=citecolor,
      }
    % Slightly bigger margins than the latex defaults
    
    \geometry{verbose,tmargin=1in,bmargin=1in,lmargin=1in,rmargin=1in}
    
    

    \begin{document}
    
    
    \maketitle
    
    

    
    \hypertarget{the-components-of-a-web-page}{%
\subsubsection{The components of a web
page}\label{the-components-of-a-web-page}}

When we visit a web page, our web browser makes a request to a web
server. This request is called a \texttt{GET} request, since we're
getting files from the server. The server then sends back files that
tell our browser how to render the page for us. The files fall into a
few main types: * HTML --- contain the main content of the page. * CSS
--- add styling to make the page look nicer. * JS --- Javascript files
add interactivity to web pages. * Images --- image formats, such as JPG
and PNG allow web pages to show pictures.

    \hypertarget{html}{%
\subsubsection{HTML}\label{html}}

\hypertarget{introduction}{%
\paragraph{1. Introduction}\label{introduction}}

HTML a markup language that tells a browser how to layout content. HTML
allows you to do similar things to what you do in a word processor like
Microsoft Word --- make text bold, create paragraphs, and so on.

HTML consists of elements called tags. The most basic tag is the
\texttt{\textless{}html\textgreater{}} tag. This tag tells the web
browser that everything inside of it is HTML.

\begin{Shaded}
\begin{Highlighting}[]
\KeywordTok{<html>}
    \KeywordTok{<head>}
    \KeywordTok{</head>}
    \KeywordTok{<body>}
        \KeywordTok{<p>}
\NormalTok{            Here's a paragraph of text!}
        \KeywordTok{</p>}
        \KeywordTok{<p>}
\NormalTok{            Here's a second paragraph of text!}
        \KeywordTok{</p>}
    \KeywordTok{</body>}
\KeywordTok{</html>}
\end{Highlighting}
\end{Shaded}

\hypertarget{heres-how-this-will-look}{%
\paragraph{Here's how this will look:}\label{heres-how-this-will-look}}

Here's a paragraph of text!

Here's a second paragraph of text!

Tags have commonly used names that depend on their position in relation
to other tags:

\begin{itemize}
\tightlist
\item
  \texttt{child} --- a child is a tag inside another tag. So the two
  \texttt{p} tags above are both children of the \texttt{body} tag.
\item
  \texttt{parent} --- a parent is the tag that another tag is inside.
  e.g.~the \texttt{html} tag is the parent of the \texttt{body} tag.
\item
  \texttt{sibiling} --- a sibiling is a tag that inside the same parent
  as another tag. e.g., \texttt{head} and \texttt{body} are siblings,
  since they're both inside \texttt{html}.
\end{itemize}

    \hypertarget{html-hyperlink}{%
\paragraph{2.HTML hyperlink}\label{html-hyperlink}}

\begin{Shaded}
\begin{Highlighting}[]
\KeywordTok{<html>}
    \KeywordTok{<head>}
    \KeywordTok{</head>}
    \KeywordTok{<body>}
        \KeywordTok{<p>}
\NormalTok{            Here's a paragraph of text!}
            \KeywordTok{<a}\OtherTok{ href=}\StringTok{"https://www.dataquest.io"}\KeywordTok{>}\NormalTok{Learn Data Science Online}\KeywordTok{</a>}
        \KeywordTok{</p>}
        \KeywordTok{<p>}
\NormalTok{            Here's a second paragraph of text!}
            \KeywordTok{<a}\OtherTok{ href=}\StringTok{"https://www.python.org"}\KeywordTok{>}\NormalTok{Python}\KeywordTok{</a>}
        \KeywordTok{</p>}
    \KeywordTok{</body>}
\KeywordTok{</html>}
\end{Highlighting}
\end{Shaded}

Output:

Here's a paragraph of text! Learn Data Science Online

Here's a second paragraph of text! Python

 \textless{}br \textgreater{} In which, \textless{}br \textgreater{}
\texttt{a} tag are links, tell the browser to render a link to another
web page. \textless{}br \textgreater{} The \texttt{href} property of the
tag determines where the link goes.\textless{}br
\textgreater{}\textless{}br \textgreater{} There are some more tags: *
\texttt{div} --- indicates a division, or area, of the page. *
\texttt{b} --- bolds any text inside. * \texttt{i} --- italicizes any
text inside. * \texttt{table} --- creates a table. * \texttt{form} ---
creates an input form.

    \hypertarget{html-classes}{%
\paragraph{3.HTML classes}\label{html-classes}}

One element can have multiple classes, and a class can be shared between
elements. Each element can only have one id, and an id can only be used
once on a page. Classes and ids are optional, and not all elements will
have them.

\begin{Shaded}
\begin{Highlighting}[]
\KeywordTok{<html>}
    \KeywordTok{<head>}
    \KeywordTok{</head>}
    \KeywordTok{<body>}
        \KeywordTok{<p}\OtherTok{ class=}\StringTok{"bold-paragraph"}\KeywordTok{>}
\NormalTok{            Here's a paragraph of text!}
            \KeywordTok{<a}\OtherTok{ href=}\StringTok{"https://www.dataquest.io"}\OtherTok{ id=}\StringTok{"learn-link"}\KeywordTok{>}\NormalTok{Learn Data Science Online}\KeywordTok{</a>}
        \KeywordTok{</p>}
        \KeywordTok{<p}\OtherTok{ class=}\StringTok{"bold-paragraph extra-large"}\KeywordTok{>}
\NormalTok{            Here's a second paragraph of text!}
            \KeywordTok{<a}\OtherTok{ href=}\StringTok{"https://www.python.org"}\OtherTok{ class=}\StringTok{"extra-large"}\KeywordTok{>}\NormalTok{Python}\KeywordTok{</a>}
        \KeywordTok{</p>}
    \KeywordTok{</body>}
\KeywordTok{</html>}
\end{Highlighting}
\end{Shaded}

Output:

Here's a paragraph of text! Learn Data Science Online

Here's a second paragraph of text! Python

    \hypertarget{request}{%
\subsubsection{Request}\label{request}}

The first thing we'll need to do to scrape a web page is to download the
page. We can download pages using the Python requests library. The
requests library will make a \texttt{GET} request to a web server, which
will download the \texttt{HTML} contents of a given web page for us.

    \begin{Verbatim}[commandchars=\\\{\}]
{\color{incolor}In [{\color{incolor}1}]:} \PY{k+kn}{import} \PY{n+nn}{requests}
        \PY{n}{page} \PY{o}{=} \PY{n}{requests}\PY{o}{.}\PY{n}{get}\PY{p}{(}\PY{l+s+s1}{\PYZsq{}}\PY{l+s+s1}{http://dataquestio.github.io/web\PYZhy{}scraping\PYZhy{}pages/simple.html}\PY{l+s+s1}{\PYZsq{}}\PY{p}{)}
        \PY{n}{page}
        
        \PY{c+c1}{\PYZsh{}Status code starting with a 2 generally indicates success, }
        \PY{c+c1}{\PYZsh{} and a code starting with a 4 or a 5 indicates an error.}
\end{Verbatim}


\begin{Verbatim}[commandchars=\\\{\}]
{\color{outcolor}Out[{\color{outcolor}1}]:} <Response [200]>
\end{Verbatim}
            
    \begin{Verbatim}[commandchars=\\\{\}]
{\color{incolor}In [{\color{incolor}2}]:} \PY{n}{page}\PY{o}{.}\PY{n}{content}  \PY{c+c1}{\PYZsh{}content of the downloaded html}
\end{Verbatim}


\begin{Verbatim}[commandchars=\\\{\}]
{\color{outcolor}Out[{\color{outcolor}2}]:} b'<!DOCTYPE html>\textbackslash{}n<html>\textbackslash{}n    <head>\textbackslash{}n        <title>A simple example page</title>\textbackslash{}n    </head>\textbackslash{}n    <body>\textbackslash{}n        <p>Here is some simple content for this page.</p>\textbackslash{}n    </body>\textbackslash{}n</html>'
\end{Verbatim}
            
    \hypertarget{beautifulsoup}{%
\subsubsection{BeautifulSoup}\label{beautifulsoup}}

html parsing tool.

    \begin{Verbatim}[commandchars=\\\{\}]
{\color{incolor}In [{\color{incolor}3}]:} \PY{k+kn}{from} \PY{n+nn}{bs4} \PY{k}{import} \PY{n}{BeautifulSoup}
        \PY{n}{soup} \PY{o}{=} \PY{n}{BeautifulSoup}\PY{p}{(}\PY{n}{page}\PY{o}{.}\PY{n}{content}\PY{p}{,}\PY{l+s+s1}{\PYZsq{}}\PY{l+s+s1}{html.parser}\PY{l+s+s1}{\PYZsq{}}\PY{p}{)}
        \PY{n}{soup}
\end{Verbatim}


\begin{Verbatim}[commandchars=\\\{\}]
{\color{outcolor}Out[{\color{outcolor}3}]:} <!DOCTYPE html>
        
        <html>
        <head>
        <title>A simple example page</title>
        </head>
        <body>
        <p>Here is some simple content for this page.</p>
        </body>
        </html>
\end{Verbatim}
            
    \begin{Verbatim}[commandchars=\\\{\}]
{\color{incolor}In [{\color{incolor}4}]:} \PY{c+c1}{\PYZsh{} format \PYZhy{} nested}
        \PY{n+nb}{print}\PY{p}{(}\PY{n}{soup}\PY{o}{.}\PY{n}{prettify}\PY{p}{(}\PY{p}{)}\PY{p}{)}
\end{Verbatim}


    \begin{Verbatim}[commandchars=\\\{\}]
<!DOCTYPE html>
<html>
 <head>
  <title>
   A simple example page
  </title>
 </head>
 <body>
  <p>
   Here is some simple content for this page.
  </p>
 </body>
</html>

    \end{Verbatim}

    \begin{Verbatim}[commandchars=\\\{\}]
{\color{incolor}In [{\color{incolor}5}]:} \PY{c+c1}{\PYZsh{} list all the elements at the top level of the page}
        \PY{n+nb}{list}\PY{p}{(}\PY{n}{soup}\PY{o}{.}\PY{n}{children}\PY{p}{)}
        \PY{c+c1}{\PYZsh{} }
\end{Verbatim}


\begin{Verbatim}[commandchars=\\\{\}]
{\color{outcolor}Out[{\color{outcolor}5}]:} ['html', '\textbackslash{}n', <html>
         <head>
         <title>A simple example page</title>
         </head>
         <body>
         <p>Here is some simple content for this page.</p>
         </body>
         </html>]
\end{Verbatim}
            
    \begin{Verbatim}[commandchars=\\\{\}]
{\color{incolor}In [{\color{incolor}6}]:} \PY{p}{[}\PY{n+nb}{type}\PY{p}{(}\PY{n}{item}\PY{p}{)} \PY{k}{for} \PY{n}{item} \PY{o+ow}{in} \PY{n+nb}{list}\PY{p}{(}\PY{n}{soup}\PY{o}{.}\PY{n}{children}\PY{p}{)}\PY{p}{]}
\end{Verbatim}


\begin{Verbatim}[commandchars=\\\{\}]
{\color{outcolor}Out[{\color{outcolor}6}]:} [bs4.element.Doctype, bs4.element.NavigableString, bs4.element.Tag]
\end{Verbatim}
            
    All of the items are \texttt{BeautifulSoup} objects: 1. \texttt{Doctype}
contains information about the type of the document 2.
\texttt{NavigableString} represent text found in the HTML document 3.
\texttt{Tag} contains nested tags.

    \begin{Verbatim}[commandchars=\\\{\}]
{\color{incolor}In [{\color{incolor}7}]:} \PY{c+c1}{\PYZsh{} select html items from soup\PYZsq{}s children:}
        \PY{n}{html} \PY{o}{=} \PY{n+nb}{list}\PY{p}{(}\PY{n}{soup}\PY{o}{.}\PY{n}{children}\PY{p}{)}\PY{p}{[}\PY{l+m+mi}{2}\PY{p}{]}
        \PY{n}{html}
\end{Verbatim}


\begin{Verbatim}[commandchars=\\\{\}]
{\color{outcolor}Out[{\color{outcolor}7}]:} <html>
        <head>
        <title>A simple example page</title>
        </head>
        <body>
        <p>Here is some simple content for this page.</p>
        </body>
        </html>
\end{Verbatim}
            
    \begin{Verbatim}[commandchars=\\\{\}]
{\color{incolor}In [{\color{incolor}8}]:} \PY{c+c1}{\PYZsh{} select body items from html\PYZsq{}s children}
        \PY{n}{body} \PY{o}{=}  \PY{n+nb}{list}\PY{p}{(}\PY{n}{html}\PY{o}{.}\PY{n}{children}\PY{p}{)}\PY{p}{[}\PY{l+m+mi}{3}\PY{p}{]}
        \PY{n}{body}
\end{Verbatim}


\begin{Verbatim}[commandchars=\\\{\}]
{\color{outcolor}Out[{\color{outcolor}8}]:} <body>
        <p>Here is some simple content for this page.</p>
        </body>
\end{Verbatim}
            
    \begin{Verbatim}[commandchars=\\\{\}]
{\color{incolor}In [{\color{incolor}9}]:} \PY{c+c1}{\PYZsh{} select p items from body\PYZsq{}s children}
        \PY{n}{p} \PY{o}{=} \PY{n+nb}{list}\PY{p}{(}\PY{n}{body}\PY{o}{.}\PY{n}{children}\PY{p}{)}\PY{p}{[}\PY{l+m+mi}{1}\PY{p}{]}
        \PY{n+nb}{print}\PY{p}{(}\PY{n}{p}\PY{o}{.}\PY{n}{get\PYZus{}text}\PY{p}{(}\PY{p}{)}\PY{p}{)}
\end{Verbatim}


    \begin{Verbatim}[commandchars=\\\{\}]
Here is some simple content for this page.

    \end{Verbatim}

    \begin{Verbatim}[commandchars=\\\{\}]
{\color{incolor}In [{\color{incolor}10}]:} \PY{c+c1}{\PYZsh{} Finding all instance of a tag at once, instead of parsing layer by layer}
         \PY{n}{soup} \PY{o}{=} \PY{n}{BeautifulSoup}\PY{p}{(}\PY{n}{page}\PY{o}{.}\PY{n}{content}\PY{p}{,}\PY{l+s+s1}{\PYZsq{}}\PY{l+s+s1}{html.parser}\PY{l+s+s1}{\PYZsq{}}\PY{p}{)}
         \PY{n}{soup}\PY{o}{.}\PY{n}{find\PYZus{}all}\PY{p}{(}\PY{l+s+s1}{\PYZsq{}}\PY{l+s+s1}{p}\PY{l+s+s1}{\PYZsq{}}\PY{p}{)}
\end{Verbatim}


\begin{Verbatim}[commandchars=\\\{\}]
{\color{outcolor}Out[{\color{outcolor}10}]:} [<p>Here is some simple content for this page.</p>]
\end{Verbatim}
            
    \begin{Verbatim}[commandchars=\\\{\}]
{\color{incolor}In [{\color{incolor}11}]:} \PY{n}{soup}\PY{o}{.}\PY{n}{find\PYZus{}all}\PY{p}{(}\PY{l+s+s1}{\PYZsq{}}\PY{l+s+s1}{p}\PY{l+s+s1}{\PYZsq{}}\PY{p}{)}\PY{p}{[}\PY{l+m+mi}{0}\PY{p}{]}\PY{o}{.}\PY{n}{get\PYZus{}text}\PY{p}{(}\PY{p}{)}
\end{Verbatim}


\begin{Verbatim}[commandchars=\\\{\}]
{\color{outcolor}Out[{\color{outcolor}11}]:} 'Here is some simple content for this page.'
\end{Verbatim}
            
    \hypertarget{searching-for-tags-by-class-and-id}{%
\subsubsection{Searching for tags by class and
id}\label{searching-for-tags-by-class-and-id}}

Classes and ids are used by CSS to determine which HTML elements to
apply certain styles to. We can also use them when scraping to specify
specific elements we want to scrape. To illustrate this principle, we'll
work with the following page:

\begin{Shaded}
\begin{Highlighting}[]
\KeywordTok{<html>}
    \KeywordTok{<head>}
        \KeywordTok{<title>}\NormalTok{A simple example page}\KeywordTok{</title>}
    \KeywordTok{</head>}
    \KeywordTok{<body>}
        \KeywordTok{<div>}
            \KeywordTok{<p}\OtherTok{ class=}\StringTok{"inner-text first-item"}\OtherTok{ id=}\StringTok{"first"}\KeywordTok{>}
\NormalTok{                First paragraph.}
            \KeywordTok{</p>}
            \KeywordTok{<p}\OtherTok{ class=}\StringTok{"inner-text"}\KeywordTok{>}
\NormalTok{                Second paragraph.}
            \KeywordTok{</p>}
        \KeywordTok{</div>}
        \KeywordTok{<p}\OtherTok{ class=}\StringTok{"outer-text first-item"}\OtherTok{ id=}\StringTok{"second"}\KeywordTok{>}
            \KeywordTok{<b>}
\NormalTok{                First outer paragraph.}
            \KeywordTok{</b>}
        \KeywordTok{</p>}
        \KeywordTok{<p}\OtherTok{ class=}\StringTok{"outer-text"}\KeywordTok{>}
            \KeywordTok{<b>}
\NormalTok{                Second outer paragraph.}
            \KeywordTok{</b>}
        \KeywordTok{</p>}
    \KeywordTok{</body>}
\KeywordTok{</html>}
\end{Highlighting}
\end{Shaded}

Output:

A simple example page

\begin{verbatim}
        <p class="inner-text first-item" id="first">
            First paragraph.
        </p>
        <p class="inner-text">
            Second paragraph.
        </p>
    </div>
    <p class="outer-text first-item" id="second">
        <b>
            First outer paragraph.
        </b>
    </p>
    <p class="outer-text">
        <b>
            Second outer paragraph.
        </b>
    </p>
</body>
\end{verbatim}

    \begin{Verbatim}[commandchars=\\\{\}]
{\color{incolor}In [{\color{incolor}12}]:} \PY{n}{page} \PY{o}{=} \PY{n}{requests}\PY{o}{.}\PY{n}{get}\PY{p}{(}\PY{l+s+s2}{\PYZdq{}}\PY{l+s+s2}{http://dataquestio.github.io/web\PYZhy{}scraping\PYZhy{}pages/ids\PYZus{}and\PYZus{}classes.html}\PY{l+s+s2}{\PYZdq{}}\PY{p}{)}
         \PY{n}{soup} \PY{o}{=} \PY{n}{BeautifulSoup}\PY{p}{(}\PY{n}{page}\PY{o}{.}\PY{n}{content}\PY{p}{,} \PY{l+s+s1}{\PYZsq{}}\PY{l+s+s1}{html.parser}\PY{l+s+s1}{\PYZsq{}}\PY{p}{)}
         \PY{c+c1}{\PYZsh{} search outer\PYZhy{}text,using class name}
         \PY{n}{soup}\PY{o}{.}\PY{n}{find\PYZus{}all}\PY{p}{(}\PY{l+s+s1}{\PYZsq{}}\PY{l+s+s1}{p}\PY{l+s+s1}{\PYZsq{}}\PY{p}{,} \PY{n}{class\PYZus{}}\PY{o}{=}\PY{l+s+s1}{\PYZsq{}}\PY{l+s+s1}{outer\PYZhy{}text}\PY{l+s+s1}{\PYZsq{}}\PY{p}{)}
\end{Verbatim}


\begin{Verbatim}[commandchars=\\\{\}]
{\color{outcolor}Out[{\color{outcolor}12}]:} [<p class="outer-text first-item" id="second">
          <b>
                          First outer paragraph.
                      </b>
          </p>, <p class="outer-text">
          <b>
                          Second outer paragraph.
                      </b>
          </p>]
\end{Verbatim}
            
    \begin{Verbatim}[commandchars=\\\{\}]
{\color{incolor}In [{\color{incolor}13}]:} \PY{c+c1}{\PYZsh{} search according to the id}
         \PY{n}{soup}\PY{o}{.}\PY{n}{find\PYZus{}all}\PY{p}{(}\PY{n+nb}{id}\PY{o}{=}\PY{l+s+s2}{\PYZdq{}}\PY{l+s+s2}{first}\PY{l+s+s2}{\PYZdq{}}\PY{p}{)}
\end{Verbatim}


\begin{Verbatim}[commandchars=\\\{\}]
{\color{outcolor}Out[{\color{outcolor}13}]:} [<p class="inner-text first-item" id="first">
                          First paragraph.
                      </p>]
\end{Verbatim}
            
    \hypertarget{searching-according-to-the-css-selectors}{%
\subsubsection{searching according to the CSS
selectors}\label{searching-according-to-the-css-selectors}}

CSS language selectors allows developers to specify HTML tags to style

\begin{itemize}
\tightlist
\item
  \texttt{p\ a} --- finds all \texttt{a} tags inside of a \texttt{p}
  tag.
\item
  \texttt{body\ p\ a} --- finds all \texttt{a} tags inside of a
  \texttt{p} tag inside of a \texttt{body} tag.
\item
  \texttt{html\ body} --- finds all \texttt{body} tags inside of an
  \texttt{html} tag.
\item
  \texttt{p.outer-text} --- finds all \texttt{p} tags with a class of
  \texttt{outer-text}.
\item
  \texttt{p\#first} --- finds all \texttt{p} tags with an id of
  \texttt{first}.
\item
  \texttt{body\ p.outer-text} --- finds any \texttt{p} tags with a class
  of \texttt{outer-text} inside of a \texttt{body} tag.
\end{itemize}

    \begin{Verbatim}[commandchars=\\\{\}]
{\color{incolor}In [{\color{incolor}14}]:} \PY{c+c1}{\PYZsh{} select p inside of div}
         \PY{n}{soup}\PY{o}{.}\PY{n}{select}\PY{p}{(}\PY{l+s+s2}{\PYZdq{}}\PY{l+s+s2}{div p}\PY{l+s+s2}{\PYZdq{}}\PY{p}{)}
\end{Verbatim}


\begin{Verbatim}[commandchars=\\\{\}]
{\color{outcolor}Out[{\color{outcolor}14}]:} [<p class="inner-text first-item" id="first">
                          First paragraph.
                      </p>, <p class="inner-text">
                          Second paragraph.
                      </p>]
\end{Verbatim}
            
    \hypertarget{case-study}{%
\subsection{Case study}\label{case-study}}

extract weather data from from the National Weather Service website.
(https://www.weather.gov), inspect elements of the web site。 In this
case, it's a \texttt{div} tag with the id \texttt{seven-day-forecast}:

\textless{}br \textgreater{} If you click around on the console, and
explore the div, you'll discover that each forecast item (like
``Tonight'', ``Thursday'', and ``Thursday Night'') is contained in a div
with the class \texttt{tombstone-container}. \textless{}br
\textgreater{}\textless{}br \textgreater{}

parsing the Page: * Download the web page cotaining the forecast *
Create a \texttt{BeautifulSoup} class the parse the page. * Find the
\texttt{div} with id \texttt{seven-day-forecast}, and assign to
\texttt{seven\_day} * Inside \texttt{seven\_day}, find each individual
forecast item. * Extract and print the first forecast item.

    \begin{Verbatim}[commandchars=\\\{\}]
{\color{incolor}In [{\color{incolor}15}]:} \PY{k+kn}{import} \PY{n+nn}{requests}
         \PY{k+kn}{from} \PY{n+nn}{bs4} \PY{k}{import} \PY{n}{BeautifulSoup}
         \PY{n}{url} \PY{o}{=} \PY{l+s+s1}{\PYZsq{}}\PY{l+s+s1}{https://forecast.weather.gov/MapClick.php?lat=47.60357000000005\PYZam{}lon=\PYZhy{}122.32944999999995\PYZsh{}.WqacHmacZdA}\PY{l+s+s1}{\PYZsq{}}
         \PY{n}{page} \PY{o}{=} \PY{n}{requests}\PY{o}{.}\PY{n}{get}\PY{p}{(}\PY{n}{url}\PY{p}{)}
         \PY{n}{soup} \PY{o}{=} \PY{n}{BeautifulSoup}\PY{p}{(}\PY{n}{page}\PY{o}{.}\PY{n}{content}\PY{p}{,}\PY{l+s+s1}{\PYZsq{}}\PY{l+s+s1}{html.parser}\PY{l+s+s1}{\PYZsq{}}\PY{p}{)}
         \PY{n}{seven\PYZus{}day} \PY{o}{=} \PY{n}{soup}\PY{o}{.}\PY{n}{find}\PY{p}{(}\PY{n+nb}{id}\PY{o}{=}\PY{l+s+s2}{\PYZdq{}}\PY{l+s+s2}{seven\PYZhy{}day\PYZhy{}forecast}\PY{l+s+s2}{\PYZdq{}}\PY{p}{)}
         \PY{n}{forecast\PYZus{}items} \PY{o}{=} \PY{n}{seven\PYZus{}day}\PY{o}{.}\PY{n}{find\PYZus{}all}\PY{p}{(}\PY{n}{class\PYZus{}}\PY{o}{=}\PY{l+s+s2}{\PYZdq{}}\PY{l+s+s2}{tombstone\PYZhy{}container}\PY{l+s+s2}{\PYZdq{}}\PY{p}{)}
         \PY{n}{tonight} \PY{o}{=} \PY{n}{forecast\PYZus{}items}\PY{p}{[}\PY{l+m+mi}{0}\PY{p}{]}
         \PY{n+nb}{print}\PY{p}{(}\PY{n}{tonight}\PY{o}{.}\PY{n}{prettify}\PY{p}{(}\PY{p}{)}\PY{p}{)}
\end{Verbatim}


    \begin{Verbatim}[commandchars=\\\{\}]
<div class="tombstone-container">
 <p class="period-name">
  Tonight
  <br/>
  <br/>
 </p>
 <p>
  <img alt="Tonight: Increasing clouds, with a low around 50. East southeast wind 12 to 17 mph decreasing to 5 to 10 mph after midnight. Winds could gust as high as 23 mph. " class="forecast-icon" src="newimages/medium/nbkn.png" title="Tonight: Increasing clouds, with a low around 50. East southeast wind 12 to 17 mph decreasing to 5 to 10 mph after midnight. Winds could gust as high as 23 mph. "/>
 </p>
 <p class="short-desc">
  Increasing
  <br/>
  Clouds
 </p>
 <p class="temp temp-low">
  Low: 50 °F
 </p>
</div>

    \end{Verbatim}

    \hypertarget{further-extraction}{%
\subsubsection{Further extraction:}\label{further-extraction}}

Analysis: inside the forcast item \texttt{tonight} is all the
information we want.There are 4 pieces of information we can extract: *
The name of the period-name forecast item - \texttt{tonight} * The
description of the conditions --- properties of \texttt{img}. * A short
description of the conditions --- in this case, \texttt{Clouds}. *
Temperature low - 50 degrees

    \begin{Verbatim}[commandchars=\\\{\}]
{\color{incolor}In [{\color{incolor}16}]:} \PY{c+c1}{\PYZsh{}further extraction:}
         \PY{n}{period} \PY{o}{=} \PY{n}{tonight}\PY{o}{.}\PY{n}{find}\PY{p}{(}\PY{n}{class\PYZus{}}\PY{o}{=}\PY{l+s+s2}{\PYZdq{}}\PY{l+s+s2}{period\PYZhy{}name}\PY{l+s+s2}{\PYZdq{}}\PY{p}{)}\PY{o}{.}\PY{n}{get\PYZus{}text}\PY{p}{(}\PY{p}{)}
         \PY{n}{short\PYZus{}desc} \PY{o}{=} \PY{n}{tonight}\PY{o}{.}\PY{n}{find}\PY{p}{(}\PY{n}{class\PYZus{}}\PY{o}{=}\PY{l+s+s2}{\PYZdq{}}\PY{l+s+s2}{short\PYZhy{}desc}\PY{l+s+s2}{\PYZdq{}}\PY{p}{)}\PY{o}{.}\PY{n}{get\PYZus{}text}\PY{p}{(}\PY{p}{)}
         \PY{n}{temp} \PY{o}{=} \PY{n}{tonight}\PY{o}{.}\PY{n}{find}\PY{p}{(}\PY{n}{class\PYZus{}}\PY{o}{=}\PY{l+s+s2}{\PYZdq{}}\PY{l+s+s2}{temp}\PY{l+s+s2}{\PYZdq{}}\PY{p}{)}\PY{o}{.}\PY{n}{get\PYZus{}text}\PY{p}{(}\PY{p}{)}
         
         \PY{n+nb}{print}\PY{p}{(}\PY{n}{period}\PY{p}{)}
         \PY{n+nb}{print}\PY{p}{(}\PY{n}{short\PYZus{}desc}\PY{p}{)}
         \PY{n+nb}{print}\PY{p}{(}\PY{n}{temp}\PY{p}{)}
\end{Verbatim}


    \begin{Verbatim}[commandchars=\\\{\}]
Tonight
IncreasingClouds
Low: 50 °F

    \end{Verbatim}

    \begin{Verbatim}[commandchars=\\\{\}]
{\color{incolor}In [{\color{incolor}17}]:} \PY{n}{img} \PY{o}{=} \PY{n}{tonight}\PY{o}{.}\PY{n}{find}\PY{p}{(}\PY{l+s+s1}{\PYZsq{}}\PY{l+s+s1}{img}\PY{l+s+s1}{\PYZsq{}}\PY{p}{)}
         \PY{n}{desc} \PY{o}{=} \PY{n}{img}\PY{p}{[}\PY{l+s+s1}{\PYZsq{}}\PY{l+s+s1}{title}\PY{l+s+s1}{\PYZsq{}}\PY{p}{]}
         \PY{n+nb}{print}\PY{p}{(}\PY{n}{desc}\PY{p}{)}
\end{Verbatim}


    \begin{Verbatim}[commandchars=\\\{\}]
Tonight: Increasing clouds, with a low around 50. East southeast wind 12 to 17 mph decreasing to 5 to 10 mph after midnight. Winds could gust as high as 23 mph. 

    \end{Verbatim}

    \hypertarget{extract-all-the-information-from-the-page}{%
\subsubsection{extract all the information from the
page}\label{extract-all-the-information-from-the-page}}

\begin{itemize}
\tightlist
\item
  Select all items with the class \texttt{period-name} inside an item
  with the class \texttt{tombstone-container} in \texttt{seven\_day}
\item
  Use a list comprehension to call the \texttt{get\_text} method on each
  \texttt{BeautifulSoup} object.
\end{itemize}

    \begin{Verbatim}[commandchars=\\\{\}]
{\color{incolor}In [{\color{incolor}18}]:} \PY{n}{period\PYZus{}tags} \PY{o}{=} \PY{n}{seven\PYZus{}day}\PY{o}{.}\PY{n}{select}\PY{p}{(}\PY{l+s+s2}{\PYZdq{}}\PY{l+s+s2}{.tombstone\PYZhy{}container .period\PYZhy{}name}\PY{l+s+s2}{\PYZdq{}}\PY{p}{)}
         \PY{n}{periods} \PY{o}{=} \PY{p}{[}\PY{n}{pt}\PY{o}{.}\PY{n}{get\PYZus{}text}\PY{p}{(}\PY{p}{)} \PY{k}{for} \PY{n}{pt} \PY{o+ow}{in} \PY{n}{period\PYZus{}tags}\PY{p}{]}
         \PY{n}{periods}
\end{Verbatim}


\begin{Verbatim}[commandchars=\\\{\}]
{\color{outcolor}Out[{\color{outcolor}18}]:} ['Tonight',
          'Tuesday',
          'TuesdayNight',
          'Wednesday',
          'WednesdayNight',
          'Thursday',
          'ThursdayNight',
          'Friday',
          'FridayNight']
\end{Verbatim}
            
    \begin{Verbatim}[commandchars=\\\{\}]
{\color{incolor}In [{\color{incolor}19}]:} \PY{n}{short\PYZus{}descs} \PY{o}{=} \PY{p}{[}\PY{n}{sd}\PY{o}{.}\PY{n}{get\PYZus{}text}\PY{p}{(}\PY{p}{)} \PY{k}{for} \PY{n}{sd} \PY{o+ow}{in} \PY{n}{seven\PYZus{}day}\PY{o}{.}\PY{n}{select}\PY{p}{(}\PY{l+s+s2}{\PYZdq{}}\PY{l+s+s2}{.tombstone\PYZhy{}container .short\PYZhy{}desc}\PY{l+s+s2}{\PYZdq{}}\PY{p}{)}\PY{p}{]}
         \PY{n}{temps} \PY{o}{=} \PY{p}{[}\PY{n}{t}\PY{o}{.}\PY{n}{get\PYZus{}text}\PY{p}{(}\PY{p}{)} \PY{k}{for} \PY{n}{t} \PY{o+ow}{in} \PY{n}{seven\PYZus{}day}\PY{o}{.}\PY{n}{select}\PY{p}{(}\PY{l+s+s2}{\PYZdq{}}\PY{l+s+s2}{.tombstone\PYZhy{}container .temp}\PY{l+s+s2}{\PYZdq{}}\PY{p}{)}\PY{p}{]}
         \PY{n}{descs} \PY{o}{=} \PY{p}{[}\PY{n}{d}\PY{p}{[}\PY{l+s+s2}{\PYZdq{}}\PY{l+s+s2}{title}\PY{l+s+s2}{\PYZdq{}}\PY{p}{]} \PY{k}{for} \PY{n}{d} \PY{o+ow}{in} \PY{n}{seven\PYZus{}day}\PY{o}{.}\PY{n}{select}\PY{p}{(}\PY{l+s+s2}{\PYZdq{}}\PY{l+s+s2}{.tombstone\PYZhy{}container img}\PY{l+s+s2}{\PYZdq{}}\PY{p}{)}\PY{p}{]}
         
         \PY{n+nb}{print}\PY{p}{(}\PY{n}{short\PYZus{}descs}\PY{p}{)}
         \PY{n+nb}{print}\PY{p}{(}\PY{n}{temps}\PY{p}{)}
         \PY{n+nb}{print}\PY{p}{(}\PY{n}{descs}\PY{p}{)}
\end{Verbatim}


    \begin{Verbatim}[commandchars=\\\{\}]
['IncreasingClouds', 'Rain andBreezy', 'Rain andBreezy', 'ShowersLikely', 'ChanceShowers', 'ChanceShowers', 'Slight ChanceShowers', 'Slight ChanceShowers', 'Slight ChanceShowers']
['Low: 50 °F', 'High: 55 °F', 'Low: 40 °F', 'High: 51 °F', 'Low: 38 °F', 'High: 53 °F', 'Low: 37 °F', 'High: 54 °F', 'Low: 40 °F']
['Tonight: Increasing clouds, with a low around 50. East southeast wind 12 to 17 mph decreasing to 5 to 10 mph after midnight. Winds could gust as high as 23 mph. ', 'Tuesday: Rain, mainly after 11am.  High near 55. Breezy, with an east southeast wind 10 to 15 mph becoming south 17 to 22 mph in the morning. Winds could gust as high as 29 mph.  Chance of precipitation is 80\%. New precipitation amounts between a tenth and quarter of an inch possible. ', 'Tuesday Night: Rain.  Low around 40. Breezy, with a west southwest wind 18 to 23 mph becoming west southwest 8 to 13 mph in the evening. Winds could gust as high as 30 mph.  Chance of precipitation is 80\%. New precipitation amounts of less than a tenth of an inch possible. ', 'Wednesday: Showers likely.  Mostly cloudy, with a high near 51. North wind 8 to 17 mph becoming southwest in the afternoon. Winds could gust as high as 23 mph.  Chance of precipitation is 60\%. New precipitation amounts between a tenth and quarter of an inch possible. ', 'Wednesday Night: A 30 percent chance of showers.  Mostly cloudy, with a low around 38. East northeast wind 7 to 10 mph becoming south southeast after midnight. ', 'Thursday: A 30 percent chance of showers.  Mostly cloudy, with a high near 53.', 'Thursday Night: A 20 percent chance of showers.  Mostly cloudy, with a low around 37.', 'Friday: A slight chance of showers.  Partly sunny, with a high near 54.', 'Friday Night: A slight chance of showers.  Mostly cloudy, with a low around 40.']

    \end{Verbatim}

    \begin{Verbatim}[commandchars=\\\{\}]
{\color{incolor}In [{\color{incolor}20}]:} \PY{c+c1}{\PYZsh{} save the data to the pandas dataframe}
         
         \PY{k+kn}{import} \PY{n+nn}{pandas} \PY{k}{as} \PY{n+nn}{pd}
         \PY{n}{weather} \PY{o}{=} \PY{n}{pd}\PY{o}{.}\PY{n}{DataFrame}\PY{p}{(}\PY{p}{\PYZob{}}
                 \PY{l+s+s2}{\PYZdq{}}\PY{l+s+s2}{period}\PY{l+s+s2}{\PYZdq{}}\PY{p}{:} \PY{n}{periods}\PY{p}{,} 
                 \PY{l+s+s2}{\PYZdq{}}\PY{l+s+s2}{short\PYZus{}desc}\PY{l+s+s2}{\PYZdq{}}\PY{p}{:} \PY{n}{short\PYZus{}descs}\PY{p}{,} 
                 \PY{l+s+s2}{\PYZdq{}}\PY{l+s+s2}{temp}\PY{l+s+s2}{\PYZdq{}}\PY{p}{:} \PY{n}{temps}\PY{p}{,} 
                 \PY{l+s+s2}{\PYZdq{}}\PY{l+s+s2}{desc}\PY{l+s+s2}{\PYZdq{}}\PY{p}{:}\PY{n}{descs}
             \PY{p}{\PYZcb{}}\PY{p}{)}
         \PY{n}{weather}
\end{Verbatim}


\begin{Verbatim}[commandchars=\\\{\}]
{\color{outcolor}Out[{\color{outcolor}20}]:}                                                 desc          period  \textbackslash{}
         0  Tonight: Increasing clouds, with a low around {\ldots}         Tonight   
         1  Tuesday: Rain, mainly after 11am.  High near 5{\ldots}         Tuesday   
         2  Tuesday Night: Rain.  Low around 40. Breezy, w{\ldots}    TuesdayNight   
         3  Wednesday: Showers likely.  Mostly cloudy, wit{\ldots}       Wednesday   
         4  Wednesday Night: A 30 percent chance of shower{\ldots}  WednesdayNight   
         5  Thursday: A 30 percent chance of showers.  Mos{\ldots}        Thursday   
         6  Thursday Night: A 20 percent chance of showers{\ldots}   ThursdayNight   
         7  Friday: A slight chance of showers.  Partly su{\ldots}          Friday   
         8  Friday Night: A slight chance of showers.  Mos{\ldots}     FridayNight   
         
                      short\_desc         temp  
         0      IncreasingClouds   Low: 50 °F  
         1        Rain andBreezy  High: 55 °F  
         2        Rain andBreezy   Low: 40 °F  
         3         ShowersLikely  High: 51 °F  
         4         ChanceShowers   Low: 38 °F  
         5         ChanceShowers  High: 53 °F  
         6  Slight ChanceShowers   Low: 37 °F  
         7  Slight ChanceShowers  High: 54 °F  
         8  Slight ChanceShowers   Low: 40 °F  
\end{Verbatim}
            
    \begin{Verbatim}[commandchars=\\\{\}]
{\color{incolor}In [{\color{incolor}21}]:} \PY{c+c1}{\PYZsh{} Data cleaning}
         \PY{n}{temp\PYZus{}nums} \PY{o}{=} \PY{n}{weather}\PY{p}{[}\PY{l+s+s2}{\PYZdq{}}\PY{l+s+s2}{temp}\PY{l+s+s2}{\PYZdq{}}\PY{p}{]}\PY{o}{.}\PY{n}{str}\PY{o}{.}\PY{n}{extract}\PY{p}{(}\PY{l+s+s2}{\PYZdq{}}\PY{l+s+s2}{(?P\PYZlt{}temp\PYZus{}num\PYZgt{}}\PY{l+s+s2}{\PYZbs{}}\PY{l+s+s2}{d+)}\PY{l+s+s2}{\PYZdq{}}\PY{p}{,} \PY{n}{expand}\PY{o}{=}\PY{k+kc}{False}\PY{p}{)}
         \PY{n}{weather}\PY{p}{[}\PY{l+s+s2}{\PYZdq{}}\PY{l+s+s2}{temp\PYZus{}num}\PY{l+s+s2}{\PYZdq{}}\PY{p}{]} \PY{o}{=} \PY{n}{temp\PYZus{}nums}\PY{o}{.}\PY{n}{astype}\PY{p}{(}\PY{l+s+s1}{\PYZsq{}}\PY{l+s+s1}{int}\PY{l+s+s1}{\PYZsq{}}\PY{p}{)}
         \PY{n}{temp\PYZus{}nums}
\end{Verbatim}


\begin{Verbatim}[commandchars=\\\{\}]
{\color{outcolor}Out[{\color{outcolor}21}]:} 0    50
         1    55
         2    40
         3    51
         4    38
         5    53
         6    37
         7    54
         8    40
         Name: temp\_num, dtype: object
\end{Verbatim}
            
    \begin{Verbatim}[commandchars=\\\{\}]
{\color{incolor}In [{\color{incolor}22}]:} \PY{c+c1}{\PYZsh{} Data filter}
         \PY{n}{is\PYZus{}night} \PY{o}{=} \PY{n}{weather}\PY{p}{[}\PY{l+s+s2}{\PYZdq{}}\PY{l+s+s2}{temp}\PY{l+s+s2}{\PYZdq{}}\PY{p}{]}\PY{o}{.}\PY{n}{str}\PY{o}{.}\PY{n}{contains}\PY{p}{(}\PY{l+s+s2}{\PYZdq{}}\PY{l+s+s2}{Low}\PY{l+s+s2}{\PYZdq{}}\PY{p}{)}
         \PY{n}{weather}\PY{p}{[}\PY{l+s+s2}{\PYZdq{}}\PY{l+s+s2}{is\PYZus{}night}\PY{l+s+s2}{\PYZdq{}}\PY{p}{]} \PY{o}{=} \PY{n}{is\PYZus{}night}
         \PY{n}{is\PYZus{}night}
\end{Verbatim}


\begin{Verbatim}[commandchars=\\\{\}]
{\color{outcolor}Out[{\color{outcolor}22}]:} 0     True
         1    False
         2     True
         3    False
         4     True
         5    False
         6     True
         7    False
         8     True
         Name: temp, dtype: bool
\end{Verbatim}
            
    \begin{Verbatim}[commandchars=\\\{\}]
{\color{incolor}In [{\color{incolor}23}]:} \PY{n}{weather}\PY{p}{[}\PY{n}{is\PYZus{}night}\PY{p}{]}
\end{Verbatim}


\begin{Verbatim}[commandchars=\\\{\}]
{\color{outcolor}Out[{\color{outcolor}23}]:}                                                 desc          period  \textbackslash{}
         0  Tonight: Increasing clouds, with a low around {\ldots}         Tonight   
         2  Tuesday Night: Rain.  Low around 40. Breezy, w{\ldots}    TuesdayNight   
         4  Wednesday Night: A 30 percent chance of shower{\ldots}  WednesdayNight   
         6  Thursday Night: A 20 percent chance of showers{\ldots}   ThursdayNight   
         8  Friday Night: A slight chance of showers.  Mos{\ldots}     FridayNight   
         
                      short\_desc        temp  temp\_num  is\_night  
         0      IncreasingClouds  Low: 50 °F        50      True  
         2        Rain andBreezy  Low: 40 °F        40      True  
         4         ChanceShowers  Low: 38 °F        38      True  
         6  Slight ChanceShowers  Low: 37 °F        37      True  
         8  Slight ChanceShowers  Low: 40 °F        40      True  
\end{Verbatim}
            

    % Add a bibliography block to the postdoc
    
    
    
    \end{document}
